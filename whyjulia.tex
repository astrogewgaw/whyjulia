\documentclass[11pt, a4paper]{tufte-handout}
\usepackage{minted}
\usepackage{ragged2e}
\usepackage{fontspec}
\usepackage{hyperref}

\setmainfont{jbm-regular.ttf}[
    Path=fonts/,
    Extension=.ttf,
    BoldFont=jbm-bold.ttf,
    ItalicFont=jbm-italic.ttf,
    BoldItalicFont=jbm-bold-italic.ttf,
]

\setmonofont{jbm-regular.ttf}[
    Path=fonts/,
    Extension=.ttf,
    BoldFont=jbm-bold.ttf,
    ItalicFont=jbm-italic.ttf,
    BoldItalicFont=jbm-bold-italic.ttf,
]

\hypersetup{colorlinks=true, linkcolor=blue, urlcolor=blue}

\renewcommand\allcapsspacing[1]{{\addfontfeature{LetterSpace=15}#1}}
\renewcommand\smallcapsspacing[1]{{\addfontfeature{LetterSpace=10}#1}}

\newcommand{\PINT}{\href{https://github.com/nanograv/PINT}{\textbf{PINT}}}
\newcommand{\Julia}{\href{https://julialang.org/}{\textbf{Julia}}}
\newcommand{\TEMPO}{\href{http://tempo.sourceforge.net/}{\textbf{TEMPO}}}
\newcommand{\PRESTO}{\href{https://github.com/scottransom/presto}{\textbf{PRESTO}}}
\newcommand{\SIGPROC}{\href{http://sigproc.sourceforge.net/}{\textbf{SIGPROC}}}
\newcommand{\RIPTIDE}{\href{https://github.com/v-morello/riptide/}{\textbf{RIPTIDE}}}
\newcommand{\Celeste}{\href{https://github.com/jeff-regier/Celeste.jl}{\textbf{Celeste}}}
\newcommand{\Tufte}{\href{https://tufte-latex.github.io/tufte-latex/}{\textbf{Tufte-\LaTeX}}}

\newcommand{\Repo}{\url{https://github.com/astrogewgaw/whyjulia}}

\date{\today}
\title{Why Julia?}
\author{
    \vspace{2mm}
    \Large Ujjwal Panda\\
    \vspace{2mm}
    \large  : astrogewgaw.com\\
    \large  : github.com/astrogewgaw\\
    \large  : ujjwalpanda97@gmail.com\\
    \vspace{2mm}
}

\begin{document}

\maketitle

\marginnote{
    \justifying
    This document uses \Tufte{}, \LaTeX{} a class inspired by the works of
    Edward Tufte. Edward Tufte is a statistician and artist, and Professor
    Emeritus of Political Science, Statistics, and Computer Science at Yale
    University. He wrote, designed, and self-published 4 classic books on data
    visualization.
}

\section{The Two-Language Problem}

\justifying
The \textbf{two-language problem} is well-known in the field of computing.
Programmers use \textbf{high-level languages} in the initial stages of
development, but have to abandon them in the favour of \textbf{low-level
languages} when deploying their code in production. But \textit{why?} The answer
is simple: the former are slow, while the latter are not. The only solution till
now has been to write 70\% - 90\% of the code in a high-level language like
\textbf{Python}, and then patch up the bottlenecks with a low-level language
like \textbf{C/C++}. Pulsar astronomy is no exception this
trend\sidenote{\justifying To take an example from our own collaboration,
    \RIPTIDE{} uses a mix of C++ and Python to offer a speedy, high-level
interface to the \textbf{F}ast \textbf{F}olding \textbf{T}ransform.}. In fact,
libraries used most often by pulsar astronomers, such as \TEMPO{}, \PRESTO{}, or
\SIGPROC{}, are written entirely in \textbf{C}. While these codes are reliable
and performant, they have proven to be difficult to maintain, extend or
document\sidenote{\justifying The \PINT{} project has also raised similar
    concerns; in fact, the name of the project is a recursive acronym for PINT
is not TEMPO3!}.

The problems above have given rise to a disturbing trend: every research group
has taken to reinventing the wheel, giving rise to a large number of tools with
an overlapping set of features. However, advances in computing offer us a
glimmer of hope.

\section{Julia: A Possible Solution?}

Since its inception in 2012, \Julia{} has promised to be the solution to the
two-language problem: a high-level language that is as performant as low-level
languages. While this may sound too good to be true at first glance, the last
ten years have since shown us otherwise. Julia is a dynamically-typed language
(like Python). However, it is the only language of that kind that has broken
through the petaFLOP barrier: in 2017, the \Celeste{} project:

\pagebreak

\begin{enumerate}
    \item Loaded \underline{$\sim$178 TB} of image data,
    \item Produced parameter estimates for \underline{188 million} stars and galaxies,
    \item Did so in a mere \underline{14.6 minutes},
    \item Achieved a speed of \underline{1.5 petaFLOPs},
    \item Using \underline{1.3 million} threads,
    \item On \underline{9300} Knights Landing (KNL) nodes
    \item At the Cori supercomputer\sidenote{The sixth fastest supercomputer in
        the world.} at NERSC\sidenote{The \textbf{N}ational \textbf{E}nergy
        \textbf{R}esearch \textbf{S}cientific \textbf{C}omputing Center
    (\textbf{NERSC}) at Lawrence Berkeley National Laboratory (Berkeley Lab).}.
\end{enumerate}

The project was written in pure Julia, and has since set an example for what can
be achieved by pure Julia used at scale. Despite showing such potential, Julia
has not gained much attention in the pulsar/FRB astronomy community. I hope to
change that via this document, by making a strong case for Julia from a
programmer's standpoint.

\subsection{Ecosystem}
\subsection{Packaging}
\subsection{Composable}
\subsection{Performance}
\subsection{Interoperability}

\end{document}
