\documentclass[12pt, a4paper]{tufte-handout}
\usepackage{minted}
\usepackage{ragged2e}
\usepackage{fontspec}
\usepackage{hyperref}
\usepackage{geometry}

\setmainfont{jbm-regular.ttf}[
    Path=fonts/,
    Extension=.ttf,
    BoldFont=jbm-bold.ttf,
    ItalicFont=jbm-italic.ttf,
    BoldItalicFont=jbm-bold-italic.ttf,
]

\setmonofont{jm-regular.ttf}[
    Path=fonts/,
    Extension=.ttf,
    BoldFont=jm-bold.ttf,
    ItalicFont=jm-italic.ttf,
    BoldItalicFont=jm-bold-italic.ttf,
]

\usemintedstyle[Julia]{gruvbox-light}
\hypersetup{colorlinks=true, linkcolor=blue, urlcolor=blue}
\renewcommand\allcapsspacing[1]{{\addfontfeature{LetterSpace=15}#1}}
\renewcommand\smallcapsspacing[1]{{\addfontfeature{LetterSpace=10}#1}}

\newcommand{\PINT}{\href{https://github.com/nanograv/PINT}{\textbf{PINT}}}
\newcommand{\Julia}{\href{https://julialang.org/}{\textbf{Julia}}}
\newcommand{\TEMPO}{\href{http://tempo.sourceforge.net/}{\textbf{TEMPO}}}
\newcommand{\PRESTO}{\href{https://github.com/scottransom/presto}{\textbf{PRESTO}}}
\newcommand{\SIGPROC}{\href{http://sigproc.sourceforge.net/}{\textbf{SIGPROC}}}
\newcommand{\RIPTIDE}{\href{https://github.com/v-morello/riptide/}{\textbf{RIPTIDE}}}
\newcommand{\Celeste}{\href{https://github.com/jeff-regier/Celeste.jl}{\textbf{Celeste}}}
\newcommand{\DeDisp}{\href{https://github.com/astrogewgaw/DeDisp.jl}{\textbf{DeDisp.jl}}}
\newcommand{\GHRSS}{\href{http://www.ncra.tifr.res.in/~bhaswati/GHRSS.html}{\textbf{GHRSS survey}}}
\newcommand{\Tufte}{\href{https://tufte-latex.github.io/tufte-latex/}{\textbf{Tufte-\LaTeX}}}
\newcommand{\AstroAccelerate}{\href{https://github.com/AstroAccelerateOrg/astro-accelerate}{\textbf{AstroAccelerate}}}

\newcommand{\Repo}{\url{https://github.com/astrogewgaw/whyjulia}}

\title{Why Julia?}
\author{Ujjwal Panda}

\begin{document}

\newgeometry{left=1.5cm, right=1.5cm, top=1.5cm, bottom=1.5cm}
\begin{titlepage}
    \vspace*{9cm}
    \Huge \textbf{Why \Julia{}?}\\
    \vspace{0.5cm}
    \LARGE Proposal to adopt \Julia{} for the \GHRSS{} codebase.\\
    \vspace{9cm}
    \LARGE \textbf{Ujjwal Panda}\\
    \vspace{0.5cm}
     : \LARGE \href{https://astrogewgaw.com}{astrogewgaw.com}\\
     : \LARGE \href{https://github.com/astrogewgaw}{github.com/astrogewgaw}\\
     : \LARGE \href{mailto:ujjwalpanda97@gmail.com}{ujjwalpanda97@gmail.com}\\
    \vspace{2cm}
    \textbf{Last updated}: \underline{\today}.
\end{titlepage}
\restoregeometry

\marginnote{
    \justifying
    This document uses \Tufte{}, \LaTeX{} a class inspired by the works of
    Edward Tufte. Edward Tufte is a statistician and artist, and Professor
    Emeritus of Political Science, Statistics, and Computer Science at Yale
    University. He wrote, designed, and self-published 4 classic books on data
    visualization, which became the basis for the \textbf{Tufte style}.
}

\justifying
As radio telescopes produce data at both unprecedented rates and resolutions,
the task of processing such data has been handed over to highly automated data
processing pipelines. Initially, these pipelines were nothing more than shell
scripts that essentially automated the same commands that would have been
entered manually by the user; now, these have become large codebases. As many
telescopes shift to conducting their searches for pulsars and radio transients
commensal to other observations, the need for processing data in real-time has
emerged; as a result, many data pipelines run entirely on GPUs or clusters of
GPUs.\\

However, the need for high performance has given rise to a disturbing trend in
the realm of pulsar astronomy software: in order to make data pipelines that are
optimised for performance, many pulsar astronomy groups have taken to writing
their entire pipelines from scratch\sidenote{\justifying A notable exception to
this is the \AstroAccelerate{} organisation, which shares optimised versions of
several pulsar data algorithms for the GPU, for use by other groups.}. While
this guarantees performance (since all algorithms are optimised to work together
and written in the same framework), it has lead to pipelines turning into large
and monolithic codebases. These become difficult to extend, maintain, or even
document.\\

The solution to these problems is not immediately apparent, since some of them
harbor ancient demons from the domain of scientific computing. One of them is
the notorious \textbf{two language problem}. Programmers start developing new
analysis packages and algorithms in languages like \textbf{Python}. These
\textbf{high level languages} tend to ease development by providing abstraction
over the nitty-gritty details of the machine\sidenote{\justifying An example of
such details is memory management; such languages abstract this away by using
\textbf{garbage collectors}, which deallocate unused memory automatically.}.
However, when deploying their code in production, they resort to \textbf{low
level languages}, such as \textbf{C}, \textbf{C++}, or \textbf{Fortran}. Since
these have access to a machine's internal mechanisms, they allow programmers to
write highly performant code. The only solution for the problem, until recently,
has been to program the bottlenecks of a algorithm in a low level language, and
then wrap the code in a high level language. Pulsar astronomy is no stranger to
this trend\sidenote{\justifying A recent example is \RIPTIDE{}, which wraps
kernels written in C++ with Python, thus providing a high-level but performant
abstraction over the \textbf{F}ast \textbf{F}olding \textbf{A}lgorithm
(\textbf{FFA})}.\\

The aim of this document is to provide a solution to some of these problems, by
introducing \Julia{} to the domain of pulsar astronomy. To the author's
knowledge, this language has not been adopted by pulsar astronomers. It will be
the author's attempt to make a case for doing so. Do note that this is a
\textit{living} document: it is meant to evolve over time, and it exists as a
repository on GitHub here: \textbf{\Repo{}}. Feel free to suggest changes and
give feedback.\\

\end{document}
